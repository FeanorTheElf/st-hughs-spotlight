\documentclass{beamer}
\usetheme{CambridgeUS}

\usepackage{tikz,pgfplots}
\usepackage{amsmath,amssymb}
\usepackage{xcolor}
\usepackage{graphicx}
\usepackage{wasysym}
\usepackage[language=english, backend=biber, style=alphabetic, sorting=nyt]{biblatex}

\title{Public Key Cryptography}
\author{Simon Pohmann}
\institute{University of Oxford}

\newcommand{\Z}{\mathbb{Z}}

\newtheorem{algorithm}{Algorithm}

\pgfplotsset{compat=1.16}
\pgfplotsset{every x tick label/.append style={font=\tiny}}
\pgfplotsset{every y tick label/.append style={font=\tiny}}

\begin{document}

\frame{\titlepage}

%\frame{
%    \frametitle{Contents}
%    \tableofcontents
%}

\begin{frame}
    \frametitle{Remember Patrick's talk...}

    \begin{center}
        \includegraphics[height = 0.7\textheight]{crypto.png}
    \end{center}
\end{frame}

\begin{frame}
    \frametitle{Where do we encounter cryptography?}
    
    \begin{center}
        \only<1>{\includegraphics[width = 0.9\textwidth]{http.png}}
        \only<2>{\includegraphics[width = 0.9\textwidth]{https.png}}
        \only<3>{\includegraphics[height = 0.7\textheight]{whatsapp_e2e.png}}
        \only<4>{\includegraphics[height = 0.9\textheight]{wlan.png}}
    \end{center}
\end{frame}

\begin{frame}
    \frametitle{Beginnings of cryptography}

    Caesar's cipher
    \begin{center}
        \includegraphics{caesar.pdf}
    \end{center}
    \pause
    \begin{itemize}
        \item Very insecure (even if shift is unknown)
        \item Symmetric cipher
    \end{itemize}
\end{frame}

\begin{frame}
    \frametitle{Symmetric cryptography}

    \begin{overlayarea}{\textwidth}{\textheight}
        \only<1>{\includegraphics[width = \textwidth]{symmetric_crypto1.pdf}}
        \only<2>{\includegraphics[width = \textwidth]{symmetric_crypto2.pdf}}
        \only<3>{\includegraphics[width = \textwidth]{symmetric_crypto3.pdf}}
        \only<4>{\includegraphics[width = \textwidth]{symmetric_crypto4.pdf}}
        \only<5>{\includegraphics[width = \textwidth]{symmetric_crypto5.pdf}}
    \end{overlayarea}
\end{frame}

\begin{frame}
    \frametitle{Asymmetric cryptography}

    \begin{overlayarea}{\textwidth}{\textheight}
        \only<1>{\includegraphics[width = \textwidth]{asymmetric_crypto1.pdf}}
        \only<2>{\includegraphics[width = \textwidth]{asymmetric_crypto2.pdf}}
        \only<3>{\includegraphics[width = \textwidth]{asymmetric_crypto3.pdf}}
        \only<4>{\includegraphics[width = \textwidth]{asymmetric_crypto4.pdf}}
    \end{overlayarea}
\end{frame}

\begin{frame}
    \frametitle{An issue in Public Key Crypto}

    \begin{itemize}
        \item Symmetric cryptography can (in principle) be ``perfectly secure''
        \item Asymmetric cryptography cannot
    \end{itemize}

    \begin{overlayarea}{\textwidth}{\textheight}
        \only<1>{\includegraphics[width = \textwidth]{attack1.pdf}}
        \only<2>{\includegraphics[width = \textwidth]{attack2.pdf}}
        \only<3>{\includegraphics[width = \textwidth]{attack3.pdf}}
    \end{overlayarea}

\end{frame}

\begin{frame}
    \frametitle{Public Key Crypto}

    \begin{center}
        Asymmetric cryptography relies on problems, that cannot be solved \textbf{efficiently} or within \textbf{reasonable time}
    \end{center}
    \pause
    \vspace{5em}
    \begin{itemize}
        \item Usually problems with mathematical structure
        \pause
        \item Currently: Prime factorization and discrete logarithm
        \pause
        \item In the future: Quantum-computer safe problems
    \end{itemize}
\end{frame}

\begin{frame}
    \frametitle{Thank you for listening!}

    \begin{center}
        \includegraphics[height = 0.6\textheight]{xkcd1.png}
        \\
        Source: \href{https://xkcd.com/}{www.xkcd.com}
    \end{center}
\end{frame}

\end{document}