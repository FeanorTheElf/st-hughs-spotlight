\documentclass{beamer}
\usetheme{CambridgeUS}

\usepackage{tikz,pgfplots}
\usepackage{amsmath,amssymb}
\usepackage{xcolor}
\usepackage{graphicx}
\usepackage{wasysym}
\usepackage[language=english, backend=biber, style=alphabetic, sorting=nyt]{biblatex}

\title{Public Key Cryptography}
\author{Simon Pohmann}
\institute{University of Oxford}

\newcommand{\Z}{\mathbb{Z}}

\newtheorem{algorithm}{Algorithm}

\pgfplotsset{compat=1.16}
\pgfplotsset{every x tick label/.append style={font=\tiny}}
\pgfplotsset{every y tick label/.append style={font=\tiny}}

\begin{document}

\begin{frame}
    \frametitle{Asymmetric Cryptography}

    \includegraphics[width = \textwidth]{asymmetric_crypto4.pdf}
\end{frame}

\begin{frame}
    \frametitle{Man-in-the-Middle Attack}

    \begin{overlayarea}{\textwidth}{\textheight}
        \only<1>{\includegraphics[width = \textwidth]{man_in_the_middle1.pdf}}
        \only<2>{\includegraphics[width = \textwidth]{man_in_the_middle2.pdf}}
        \only<3>{\includegraphics[width = \textwidth]{man_in_the_middle3.pdf}}
        \only<4>{\includegraphics[width = \textwidth]{man_in_the_middle4.pdf}}
        \only<5>{\includegraphics[width = \textwidth]{man_in_the_middle5.pdf}}
    \end{overlayarea}
\end{frame}

\begin{frame}
    \frametitle{Authentication}

    \begin{itemize}
        \item To prevent Man-in-the-Middle attacks, Alice must verify that the message sender is Bob
        \pause
        \item Use of ``Digital Signatures''
    \end{itemize}
    \pause
    \begin{block}{Digital Signature}
        Like Public Key Encryption, but the other way round:
        \pause
        \begin{itemize}
            \item Using the Private Key, one can create a signature
            \pause
            \item Using the Public Key, one can verify a signature
        \end{itemize}
    \end{block}

\end{frame}

\begin{frame}
    \frametitle{Certification Authorities}

    \begin{block}{Digital Signature}
        Like Public Key Encryption, but the other way round:
        \begin{itemize}
            \item Using the Private Key, one can create a signature
            \item Using the Public Key, one can verify a signature
        \end{itemize}
    \end{block}
    \pause
    \begin{center}
        To verify Bob's signature, Alice still needs his public key!
    \end{center}
    \pause
    \begin{itemize}
        \item Instead use a Certification Authority that certifies that Bob's Public Key belongs to Bob
        \pause
        \item Everyone ``knows'' the Public Keys of the Certification Authorities
    \end{itemize}
\end{frame}

\begin{frame}
    \frametitle{Certification Authorities}

    \begin{overlayarea}{\textwidth}{\textheight}
        \only<1>{\includegraphics[width = \textwidth]{certification_authority1.pdf}}
        \only<2>{\includegraphics[width = \textwidth]{certification_authority2.pdf}}
        \only<3>{\includegraphics[width = \textwidth]{certification_authority3.pdf}}
        \only<4>{\includegraphics[width = \textwidth]{certification_authority4.pdf}}
        \only<5>{\includegraphics[width = \textwidth]{certification_authority5.pdf}}
    \end{overlayarea}
\end{frame}

\begin{frame}
    \frametitle{Certification Authorities}

    \includegraphics[width = \textwidth]{https.png}
\end{frame}

\begin{frame}
    \frametitle{Certification Authorities}

    \begin{center}
        \includegraphics[height = \textheight]{wlan.png}
    \end{center}
\end{frame}

\begin{frame}
    \frametitle{Thank you for listening!}

    \begin{center}
        \includegraphics[height = 0.6\textheight]{xkcd2.png}
        \\
        Source: \href{https://xkcd.com/}{www.xkcd.com}
    \end{center}
\end{frame}

\end{document}